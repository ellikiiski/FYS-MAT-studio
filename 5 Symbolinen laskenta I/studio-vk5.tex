\documentclass[a4paper,11pt]{article}

\usepackage{amsmath}
\usepackage{amssymb}
\usepackage{gensymb}
\usepackage[utf8]{inputenc}
\usepackage[T1]{fontenc}
\usepackage{parskip}
\usepackage{graphicx}
\usepackage{epstopdf}
\usepackage[finnish]{babel}

\begin{document}

{
\thispagestyle{empty}

{\large
Aalto Yliopisto
\par
SCI-C0200 - Fysiikan ja matematiikan menetelmien studio
}

\vspace{7cm}

{\huge \bf
Tietokoneharjoitus 5: 
\par
Symbolinen laskenta I}

\vspace{2cm}

{\Large Elli Kiiski}

\vfill

\clearpage

\tableofcontents

\clearpage

\section{Tehtävä A: Mathematica-pelleilyä}

\subsection{Komentoja}

Komemto \texttt{Series[Sin[x],{x,0,3}]} antaa tulostuksen $x-\frac{x^3}{6}+O[x]^4$, eli sinifunktion kolmannen asteen Taylorin polynomin. Kun komento laitetaan \texttt{Normal[]}-komennon sisään, tulostuu $x-\frac{x^3}{6}$ eli jäännöstermi jää pois.

Piin likiarvo kahdenkymmenen merkitsevän numeron tarkkuudella saadaan komennolla \texttt{N[Pi,20]}. Ja sehän on $\pi \approx 3.1415926535897932385$.

\subsection{Funktion nollakohta}

Halutaan laskea funktion $f(x)=x^2+3x-5$ nollakohdat. Määritellään ensin kyseinen funktio koodirivillä \texttt{f[x\_] = x\^{}2 + 3*x - 5} ja tarkaistaan nollakohta komennolla \texttt{Solve[f[x] == 0, x]}. Näin saadaan näppärästi tulokseksi nollakohdat $x=\frac{1}{2}(-3\pm\sqrt{29})$.

\subsection{Kuvaajia}

Piirretään kuvaajat sinifunktiosta ja sen viidennen asteen Taylorin polynomista välillä $[-2\pi,2\pi]$ samaan kuvaan. Se on onnistuu kuvan \ref{fig:sinitaylor} näyttämällä tavalla.

\begin{figure}[!htb]
    \centering
    \includegraphics[width =100mm]{kuva1-sintaylor.eps}
    \caption{Sinifunktio ja sen viidennen asteen Taylorin polynomi nollassa.}
    \label{fig:sinitaylor}
\end{figure}

\clearpage

\section{Tehtävä B: EOQ-malli}

Tutkitaan tuotteen varastointikustannuksia EOQ-mallilla. Mallissa $D$ kuvaa tuotteen kysynytää vuodessa ja $Q$ varastoon yhdessä tilauserässä tilattavien tuotteisen määrää. Lisäksi $C_1$ on varaston täydentämisestä koituva tilauskustannus ja $C_2$ varastointikustannus per varastoitava tuote per vuosi.

\subsection{Kokonaiskustannukset}

Tavoitteena on luonnollisesti minimoida tuotteen varastoinnin vuosittaiset kokonaiskustannukset, jotka voidaan ilmaista kaavalla
\begin{equation*}
    C_{total}= C_1 \frac{D}{Q} + C_2 \frac{Q}{2},
\end{equation*}
missä $ \frac{D}{Q}$ kuvaa tilausten lukumäärää vuodessa ja $\frac{Q}{2}$ keskimääräistä varastotilannetta (olettaen, että kysyntä on tasaista vuoden mittaan).

\subsection{Optimointi}

Selvitetään sitten, miten kannattaisi valita tilauskoko $Q$ ja tilauskertojen lukumäärä, jotta kokonaiskustannukset jäisivät mahdollisimman pieniksi.

Ratkaistaan siis funktion $C_{total}(Q)$ minimi Mathematican avulla. Määritellään ensin funktio rivillä \texttt{Ctot[q\_] = c1*(d/q) + c2*(q/2);} ja derivoidaan funktio sitten muuttujan $Q$ suhteen komennolla \texttt{D[Ctot[q], q]}. Ratkaistaan derivaattafunktion positiivinen nollakohta ja saadaan
\begin{align}
    0 & = C_{total}'(Q) = \frac{C_2}{2}-\frac{C_1 D}{Q^2}\\
    Q & = \sqrt{\frac{2DC_1}{C_2}},
\end{align}
eli optimaalisin tilauserän koko on $Q^*=\sqrt{\frac{2DC_1}{C_2}}$. Tilauskertojen lukumäärä on tällöin $\frac{D}{Q^*}=\sqrt{\frac{DC_2}{2C_1}}$.

Siis optimoidulla tilauskoolla kokonaiskustannusten suuruudeksi saadaan
\begin{equation}
    C_{total} = C_1 \frac{D}{Q^*} + C_2 \frac{Q^*}{2} = C_1 \sqrt{\frac{DC_2}{2C_1}} + C_2 \frac{\sqrt{\frac{2DC_1}{C_2}}}{2} = \sqrt{2DC_1C_2}.
\end{equation}

\clearpage

\section{Tehtävä C: Rosenbrockin banaanilaakso}

Tarkastellaan erästä Rosenbrockin banaanilaksifunktioksikin kutsuttua optimoinnin testauksessa yleisesti käytettyä funktiota
\begin{equation}
    f(x,y)=100(y-x^2)^2+(a-x)^2,
\end{equation}
missä $a$ on vakio.

Funktion $f$ minimi $f(a,a^2)=0$ löydetään näppärästi Mathematicalla seuraavasti:
\begin{figure}[!htb]
    \includegraphics[width =80mm]{kuva2-banaanimin.eps}
\end{figure}

Kuvassa \ref{fig:banana2} funktion $f$ graafi vakion arvolla $a=1$ minimipisteen $(1,1)$ läheisyydessä.
\begin{figure}[!htb]
    \centering
    \includegraphics[width =110mm]{kuva3-banaanigraafi.eps}
    \caption{Banaanilaaksofunktion $f$ graafi vakion arvolla $a=1$.}
    \label{fig:banana2}
\end{figure}

\section{Kotitehtävä: Kokonaisdifferentiaalin laskeminen}

Fysikaalinen suure $F$, joka ei ole suoraan mitattavissa, voidaan määrittää mittaamalla siihen vaikuttavien suureiden $x_1,x_2,..,x_n$ arvot ja laskea sille arvo kaavalla $F(x_1,x_2,...,x_n)$. Virherajat saadulle suureen $F$ arvolle saadaan puolestaan kokonaisdifferentiaalista
\begin{align}
    \Delta F & = \left\lvert \frac{\partial F}{\partial x_1} \right\rvert \Delta x_1 + \left\lvert \frac{\partial F}{\partial x_2} \right\rvert \Delta x_2 + ... + \left\lvert \frac{\partial F}{\partial x_n} \right\rvert \Delta x_n\\
    & = \lvert \nabla F \rvert \cdot (\Delta x_1, \Delta x_2, ..., \Delta x_n),
\end{align}
missä $\Delta x_i$ on mittauksen $x_i$ keskivirhe.

\subsection{Polttovälin virhearvio}
\label{polttovali}

Linssin polttoväli $f$ saadaan kaavasta $f=\frac{ab}{a+b}$, missä $a$ on esineen etäisyys ja $b$ kuvan etäisyys linssistä. Kuvasta \ref{fig:polttov} löytyy Mathematica-koodi, jolla seuraavat tulokset on laskettu.

Polttovälin virhearvion lausekke on
\begin{equation}
    \Delta f = \Delta a \left\lvert \frac{b^2}{(a+b)^2} \right\rvert + \Delta b \left\lvert \frac{a^2}{(a+b)^2} \right\rvert.
\end{equation}

Olkoon nyt $a=85\pm1$ ja $b=196\pm2$. Tällöin polttoväliksi ja virherajoiksi saadaan
\begin{equation}
    f = \frac{ab}{a+b} = \frac{16660}{281} \approx 59.29
\end{equation}
sekä
\begin{equation}
    [f_{min},f_{max}] = [f-\Delta f, f+\Delta f] = \left[\frac{4628594}{78961}, \frac{4734326}{78961}\right] \approx [58.62, 59.96].
\end{equation}

\begin{figure}
    \centering
    \includegraphics[width =90mm]{kuva4-polttov.eps}
    \caption{Osion \ref{polttovali} Mathematica-koodi.}
    \label{fig:polttov}
\end{figure}

\subsection{Ympyräsektoritontin pinta-ala}
\label{sektoritontti}

Halutaan mitata ympyräsektorin muotoisen tontin pinta-ala $0.5\,m^2$ tarkkuudella. Kulma pystytään mittaamaan luotettavasti sadasosa-asteen tarkkuudella, joten tehtävänä on selvittää, miten tarkasti säde on pystyttävä mittaamaan, jotta päästään toivottuun pinta-alan tarkkuuteen.

Pinta-ala saadaan lausekkeesta $A(r,\phi)=\frac{\phi}{2\pi}\pi r^2=\frac{1}{2}\phi r^2$, missä $r$ on ympyrän säde ja $\phi$ sektorin keskuskulma.

Muodostetaan kokonaisdifferentiaali eli pinta-alan virhetermin lauseke ja ratkaistaan siitä analyyttinen lauseke virheelle $\Delta r$:
\begin{align}
    \Delta A & = \Delta r \lvert\phi r\rvert + \frac{1}{2}\, \Delta \phi\, r^2\\
    \Delta r & = \frac{2\,\Delta A-\Delta\phi\, r^2}{2\lvert\phi r\rvert}.
\end{align}
Sijoitetaan nyt tehtävänannon arvot $\Delta A \leq 0.5\,m^2 = \frac{1}{2}m^2$ ja $\Delta\phi = 0.01\degree = \frac{\pi}{18000}$, jolloin saadaan
\begin{equation}
    \Delta r \leq \frac{18000-\pi r^2}{36000 \lvert \phi r \rvert}\,m.
\end{equation}
Säteen mittaustarkkuus on riippuu siis itse säteestä, eli absoluuttisten virherajojen laskemiseksi tarvitaan tieto säteen pituudesta.

Tarkastellaan esimerkiksi tilannetta, jossa säde $r\approx 50\,m$ ja kulma $\phi \approx \frac{2\pi}{3}$ sekä pinta-alan ja kulman virheet edelleen $\Delta A \leq \frac{1}{2}\,m^2$ ja $\Delta\phi = \frac{\pi}{18000}$. Nyt voidaan laskea yläraja virheelle $\Delta r$:
\begin{equation}
    \Delta r \leq \frac{3}{200\pi}-\frac{1}{480}\,m \approx 0.0027\,m = 2,7\,mm.
\end{equation}
Siis $50$ metrin matkalla mittaus ei saa heittää kolmea milliäkään, jos alle puolen neliömetrin pinta-alavirheessä halutaan pysyä.

Yllä olevissa laskuissa käytetty Mathematica-koodi löytyy kuvasta \ref{fig:tontti}.

\begin{figure}
    \centering
    \includegraphics[width =120mm]{kuva5-sektoritontti.eps}
    \caption{Osion \ref{sektoritontti} Mathematica-koodi.}
    \label{fig:tontti}
\end{figure}

\end{document}