\documentclass[a4paper,11pt]{article}

\usepackage{amsmath}
\usepackage{amssymb}
\usepackage{gensymb}
\usepackage[utf8]{inputenc}
\usepackage[T1]{fontenc}
\usepackage{parskip}
\usepackage{graphicx}
\usepackage{epstopdf}
\usepackage[finnish]{babel}
\usepackage{appendix}

\usepackage{color}
\definecolor{green2}{rgb}{0,0.4,0}
\usepackage{listings}
\lstset{frame=tb,
  language=MATLAB,
  aboveskip=3mm,
  belowskip=2mm,
  showstringspaces=false,
  columns=flexible,
  basicstyle={\small\ttfamily},
  numbers=none,
  commentstyle=\color{green2},
  breaklines=false,
  breakatwhitespace=false,
  tabsize=3
}

\begin{document}

{
\thispagestyle{empty}

{\large
Aalto Yliopisto
\par
SCI-C0200 - Fysiikan ja matematiikan menetelmien studio
}

\vspace{7cm}

{\huge \bf
Fysiikan harjoitus 2: 
\par
Vaihtovirtapiiri}

\vspace{2cm}

{\Large Elli Kiiski}

\clearpage

\tableofcontents

\clearpage

\section{Esitehtävät}

\subsection{Resonanssitaajuus}

\textit{Mitä tarkoitetaan vaihtovirtapiirin resonanssitaajuudella?}

Resonanssitaajuus on se vaihtovirtapiirin taajuus, jolla virta ja jännite ovat samassa vaiheessa keskenään. Tällöin myös virran amplitudi on maksimissaan.

\subsection{Oskilloskooppi}

\textit{Miten oskilloskoopilla voidaan tutkia RLC-piirin yli olevaa jännitettä ja siinä kulkevaa virtaa?}

Oskilloskooppi voidaan kytkeä mittaamaan jännitettä suoraan jännityslähteestä tai vastuksen yli. Virta puolestaa saadaan mitatauksi välillisesti jännitehäviöstä vastuksen yli laskemalla Ohmin lain avulla.

\begin{figure}
    \centering
    \includegraphics[width =60mm]{kuva1-rlcpiiri.eps}
    \caption{Mittauksissa käytettävän laitteiston piirikaavio. Jännitemittarina toimii oskilloskooppi.}
    \label{fig:rlcpiiri}
\end{figure}

\section{Mittauksia}

\subsection{Jännite ja virta vaihtovirtapiirissä}

Suoritetaan mittauksia oskilloskoopilla kuvan \ref{fig:rlcpiiri} mukaisessa vaihtovirtapiirissä, jossa $R=1\,k\Omega$, $L=68\,mH$ ja $C=0,1\,\mu F$. Tutkitaan vaihtovirtapiirin käyttäytymistä sen resonanssitaajuuden ympäristössä. Erityisesti mitataan virran amplitudia sekä vaihe-eroa taajuuden funktiona.

\subsubsection{Hypoteesi}

Arvelen mittausten tuottavan jotakuinkin seuraavanlaisten kuvaajien mukaisia tuloksia.

\begin{figure}[!htb]
    \centering
    \includegraphics[width =90mm]{kuva2-kuvaajathyp.eps}
\end{figure}

Kuvaajien täsmällistä muotoa tärkeämpiä seikkoja ovat ne, että virran amplitudi saavuttaa huippunsa resonanssitaajuudella ja vaihe-ero puolestaan on tällöin nolla.

\subsubsection{Mittaustulokset}
\label{mittitti}

Mittauksisa havaittiin virran amplitudin kasvavan taajuuden kasvaessa, kunnes se noin $1,9\,kHz$ taajuudella käänytyi laskuun. Vaihe-ero on nollassa juurikin samalla tajuudella. Näiltä osin hypoteesi piti hyvin paikkansa.

Kuitenkaan vaihe-eron kuvaajaa hypoteesiin laatiessa en tullut huomioineeksi, että resonanssitaajuutta alhaalta päin lähestyttäessä vaihe-ero on tietenkin negatiivinen, ja mittausten mukaan se onkin pienillä taajuuksilla lähellä $-90^{\circ}$ ja lähenee nollaa resonanssitaajuutta lähestyttäessä. Vastaavasti taajuuden kasvaessa vaihe-ero lähenee $+90^{\circ}$.

Mitataan vaihe-ero ja jännite vastuksen yli $34$ eri tajuudella, ja muodostetaan Ohmin lain avulla laskemalla taulukon \ref{tab:virrir} data, josta voidaan piirtää kuvan \ref{fig:kuvaajat} kuvaajat. (Tässä käytetty MATLAB-koodi löytyy tarvittaessa liitteestä \ref{koodi}.)

Kun verrataan hypoteesin kuvaajia näihin mittaustuloksista piirrettyihin kuvaajiin, arvelut tosiaankin osuivat jotakuinkin oikeaan, jos huomioidaan, että vaihe-eron kuvaaja hypoteesissa onkin ikäänkuin itseisarvo vaihe-erosta. Tietenkin myös virran amplitudin kuvaaja on siitä vinksallaan, että sen pitäisi pienemmillä taajuuksilla olla matalampi kuin suurilla.

\begin{table}[]
    \centering
    \begin{tabular}{|r|c|l|}
    \hline
    $f$ ($Hz$)  & $I$ ($mA$) & $\Phi$ ($^{\circ}$)\\
    \hline
    50 & 0,7 & 87\\
    100 & 1,4 & -85\\
    200 & 2,7 & -80\\
    300 & 4,0 & -75\\
    400 & 5,3 & -72\\
    500 & 6,6 & -67\\
    700 & 9,0 & -57\\
    800 & 10,3 & -51,5\\
    900 & 11,3 & -47,8\\
    1000 & 12,4 & -42\\
    1100 & 13,3 & -37\\
    1200 & 14,2 & -31\\
    1300 & 14,8 & -27,3\\
    1400 & 15,4 & -21,5\\
    1500 & 15,9 & -16,5\\
    1600 & 16,1 & -12,5\\
    1700 & 16,6 & -7\\
    \textbf{1847} & \textbf{16,5} & \textbf{0}\\
    1900 & 16,5 & 2\\
    2000 & 16,5 & 6\\
    2200 & 16,3 & 12\\
    2300 & 15,9 & 16\\
    2400 & 15,7 & 18\\
    2600 & 15,1 & 24\\
    2800 & 14,5 & 29,5\\
    3000 & 14,0 & 34\\
    3500 & 12,6 & 42\\
    4500 & 10,3 & 55\\
    6000 & 8,0 & 63\\
    8000 & 6,0 & 70\\
    10000 & 5,0 & 74\\
    13000 & 3,9 & 78\\
    15000 & 3,4 & 80\\
    20000 & 2,4 & 83\\
    \hline
    \end{tabular}
    \caption{Oskilloskoopilla mitattu vaihe-ero $\Phi$ sekä jännitehäviöstä laskettu sähkövirran amplitudin peak-to-peak arvo $I$ taajuuden $f$ mukaan. Resonanassitajuus on mittausten mukaan $1847\,Hz$ ja virran amplitudin maksimi $16,5\,mA$. Arvioidut virheet ovat jännittellee $\pm 0,1\,V$, vaihe-erolle $\pm 1^{\circ}$ ja resonanssitaajuudelle $\pm 5\,Hz$.}
    \label{tab:virrir}
\end{table}

\begin{figure}
    \centering
    \includegraphics[width =110mm]{kuva3-kuvaajatmit.eps}
    \caption{Mittausten tuloksena saadut kuvaajat.}
    \label{fig:kuvaajat}
\end{figure}

Teoreettinen resonanssitaajuus olisi kaavan mukaan
\begin{equation}
\label{resot}
    f_0=\frac{1}{2\pi\sqrt{LC}}=\frac{1}{2\pi\sqrt{0,068\,H \cdot 0,0000001\,F}} \approx 1930\,Hz\,,
\end{equation}
mikä heittää jonkin verran mitatusta resonanssitaajuuden arvosta $1847$. Ero on itse asiassa suurempi kuin oletettu mittausvirhe $\pm 5\,Hz$, joten jossain kohtaa mittauksia tai laskelmia on täytynyt sattua jokin yllättävämpi virhe.

\subsection{Teho vaihtovirtapiirissä}

Tarkastellaan edelleen kuvan \ref{fig:rlcpiiri} vaihtovirtapiiriä. Mitataan oskilloskoopilla sekä koko virtapiirin jännitteen että vastuksen yli olevan jännitteen tehollisarvot ja vaihe-ero sekä lasketaan niiden avulla näennäis- ja pätöteho. Mitataan myös jännitteiden aikakeskiarvo ja lasketaan sen avulla keskimääräinen teho. Vertaillaan saatuja tehoja.

\subsubsection{Mittaustulokset}

Saadaan seuraavat mittaustulokset:
\begin{itemize}
    \item Lähdejännitteen tehollisarvo: $U_0=7,28\,V$
    \item Jännitteen tehollisarvo vastuksen yli: $U_R=1.89\,V$
    \item Vaihe-ero: $\Phi = 61^{\circ}$
    \item Jännitteiden aikakeskiarvo: $U_0 \cdot U_R = 10,0\,V^2$.
\end{itemize}

Näiden avulla saadaan laskettua
\begin{itemize}
    \item Näennäisteho: $P_S=U_0 \cdot \frac{U_R}{R} = 7,28\,V \frac{1.89\,V}{1000\,\Omega} \approx 0,014\,W$
    \item Pätöteho: $P_P=P_S \cos{\Phi} \approx 0,014\,W \cdot 0,48 \approx 0,0067\,W$
    \item Keskiteho: $P_K = \frac{U_0 \cdot U_R}{R} \approx \frac{10,0\,V^2}{1000\,\Omega} \approx 0,01\,W$.
\end{itemize}

Saadut arvot eri tehoille vaikuttavat järkeviltä. Näennäisteho on suurin niin kuin pitääkin ja pätöteho huomattavasti pienempi, kuten voi niinkin suurella vaihe-erolla kuin $61^{\circ}$ odottaa. Kuulostaa myös uskottavalta, että keskimääräinen teho, joka ei ota huomioon vaihe-eroa, on suurempi kuin saatu pätöteho, jossa vaihe-ero on lähempänä $90$ astetta kuin nollaa.

\section{Kysymys sähkökondensaattorista}

\textit{Monissa elektroniikan sovellutuksissa käytetään RLC-piirejä, joissa on mukana säätökondensaattori. Mitä säätökondensaattorin käytöllä voidaan saavuttaa?}

Säätökondensaattorilla voidaan säätää vaihtovirtapiirin kapasitanssia. Näin ollen valitsemalla sen avulla sopiva kapasitanssi, voidaan piirin resonanssitaajuus asettaa halutuksi yhtälön (\ref{resot}) avulla.

\clearpage

\section{Liitteet}

\subsection{MATLAB-koodi}
\label{koodi}

Osion \ref{mittitti} laskelmissa käytetty MATLAB-koodi:
\begin{lstlisting}
% Alustetaan alkuarvot
R = 1000;
L = 0.068;
C = 0.0000001;
% Alustetaan mittaustulokset
F = xlsread('vaihtovirtapiiri.xlsx', 'A2:A35');
U = xlsread('vaihtovirtapiiri.xlsx', 'B2:B35');
Fii = xlsread('vaihtovirtapiiri.xlsx', 'C2:C35');
% Lasketaan virran amplitudi Ohmin lailla
I = U/R;
% Virta milliampeereina
I*1000

% RESONASSITAAJUUS
% mitattu
zeroFii = F(find(Fii(:)==0))
% laskennallinen
F0 = 1/(2*pi*sqrt(L*C))

% Plotataan virran amplitudi ja vaihe-ero taajuuden funktiona
% omiin kuvaajiinsa allekkain ja merkitaan mitattu resonanssitaajuus
figure
subplot(2,1,1)
plot(F,I);
xline(zeroFii);
title('Sahkovirran amplitudi taajuuden funktiona')
xlabel('f (Hz)')
ylabel('I (A)')
subplot(2,1,2)
plot(F,Fii);
xline(zeroFii);
title('Vaihe-ero taajuuden funktiona')
xlabel('f (Hz)')
ylabel('\Phi (aste)')
\end{lstlisting}

\end{document}